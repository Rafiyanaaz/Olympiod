\documentclass[12pt,-letter paper]{article}
\usepackage{siunitx}
\usepackage{setspace}
\usepackage{gensymb}
\usepackage{xcolor}
\usepackage{caption}
%\usepackage{subcaption}
\doublespacing
\singlespacing
\usepackage[none]{hyphenat}
\usepackage{amssymb}
\usepackage{relsize}
\usepackage[cmex10]{amsmath}
\usepackage{mathtools}
\usepackage{amsmath}
\usepackage{commath}
\usepackage{amsthm}
\interdisplaylinepenalty=2500
%\savesymbol{iint}
\usepackage{txfonts}
%\restoresymbol{TXF}{iint}
\usepackage{wasysym}
\usepackage{amsthm}
\usepackage{mathrsfs}
\usepackage{txfonts}
\let\vec\mathbf{}
\usepackage{stfloats}
\usepackage{float}
\usepackage{cite}
\usepackage{cases}
\usepackage{subfig}
%\usepackage{xtab}
\usepackage{longtable}
\usepackage{multirow}
%\usepackage{algorithm}
\usepackage{amssymb}
%\usepackage{algpseudocode}
\usepackage{enumitem}
\usepackage{mathtools}
%\usepackage{eenrc}
%\usepackage[framemethod=tikz]{mdframed}
\usepackage{listings}
%\usepackage{listings}
\usepackage[latin1]{inputenc}
%%\usepackage{color}{   
%%\usepackage{lscape}
\usepackage{textcomp}
\usepackage{titling}
\usepackage{hyperref}
%\usepackage{fulbigskip}   
\usepackage{tikz}
\usepackage{graphicx}
\lstset{
  frame=single,
  breaklines=true
}
\let\vec\mathbf{}
\usepackage{enumitem}
\usepackage{graphicx}
\usepackage{siunitx}
\let\vec\mathbf{}
\usepackage{enumitem}
\usepackage{graphicx}
\usepackage{enumitem}
\usepackage{tfrupee}
\usepackage{amsmath}
\usepackage{amssymb}
\usepackage{mwe} % for blindtext and example-image-a in example
\usepackage{wrapfig}
\graphicspath{{figs/}}
\providecommand{\mydet}[1]{\ensuremath{\begin{vmatrix}#1\end{vmatrix}}}
\providecommand{\myvec}[1]{\ensuremath{\begin{bmatrix}#1\end{bmatrix}}}
\providecommand{\cbrak}[1]{\ensuremath{\left\{#1\right\}}}
\providecommand{\brak}[1]{\ensuremath{\left(#1\right)}}
\begin{document}                                            \begin{enumerate}
   \item Six points are chosen on the sides of an equilateral triangle $ABC$: $A_1$,$A_2$ on $BC$,$B_1$,$B_2$ on $CA$ and $C_1$,$C_2$ on $AB$, such that they are the vertices of a convex hexagon $A_1A_2$ $B_1B_2$ $C_1C_2$ with equal side lengths.Prove that the line $A_1B_2$,$B_1C_2$ and $C_1A_2$ are concurrent.
   \item Let $a_1$,$a_2$,.... be a sequence of integers with infinitely many positive and negative terms. Suppose that for every positive integer n the numbers $a_1$,$a_2$....$a_n$ leave n different remainders upon division by n. Prove that every integer occurs exactly once in the sequence $a_1$,$a_2$.....
   \item prove that $x,y,z$ be three positve real such that $xyz$ $\geq{1}$.Prove that 
\begin{align*}
\frac{x^5-x^2}{x^5+y^2+z^2} + \frac{y^5-y^2}{x^2+y^5+z^2} + \frac{z^5-z^2}{x^2+y^2+z^5} \geq{0}
\end{align*}
   \item Determine all positive integers relatively prime to all the terms. of the infinite sequence
\begin{align*}
a_n={2^n}+{3^n}+{6^n},n\geq{1}.
\end{align*}
    \item Let $ABCD$ be a fixed convex quadrilateral with $BC$ = $DA$ and $BC$ not parallel with $DA$. Let two variable points $E$ and $F$ lie of the sides $BC$ and $DA$, respectively and satisfy $BEDF$. The lines $AC$ and $BD$ meet at $P$,the lines $BD$ and $EF$ meet at $Q$, the lines $EF$ and $AC$ meet at $R$.
Prove that the circumcircles of the triangles $PQR$, as $E$ and $F$ vary, have a common point other than $P$.
   \item In a mathematical competition, in which $6$ problems were posed to the participants, every two of these problems were solved by more than $\frac{2}{5}$ of the contestants. Moreover, no contestant solved all the $6$ problems. Show that there are at least $2$ contestants who solved exactly $5$ problems each.
	   
\end{enumerate}
\end{document}
